\PassOptionsToPackage{dvipsnames,svgnames,x11names}{xcolor}
\documentclass[landscape,a0paper,fontscale=0.312]{baposter}

% For graphs
\usepackage{graphicx}

\usepackage{array}
\usepackage{booktabs}
\usepackage{eso-pic}
\usepackage{layout}
\usepackage{fancybox}


\usepackage{calc}
\usepackage{amsmath}
\usepackage{amssymb}
\usepackage{relsize}
\usepackage{multirow}
\usepackage{rotating}
\usepackage{bm}
\usepackage{url}
\usepackage{xfrac}
\usepackage{natbib}
\usepackage{mathtools}
\usepackage{cancel}
\usepackage{paralist}
\usepackage{nicefrac}
\usepackage[export]{adjustbox} % loads also graphicx
\usepackage{caption}
\usepackage{xfrac}

% \usepackage[centering,includeheadfoot,margin=2cm]{geometry}

\usepackage{multicol}

% \usepackage[linesnumbered,ruled,vlined,noend]{algorithm2e}
% % \usepackage{algorithmicx}
% % \usepackage{algpseudocode}
% \newlength\figureheight
% \newlength\figurewidth
% \setlength{\algomargin}{2em}
% \SetKwComment{Comment}{$\blacktriangleright$\ }{}
% 
% \providecommand{\SetAlgoLined}{\SetLine}
% \providecommand{\DontPrintSemicolon}{\dontprintsemicolon}

\newcommand{\figurewidth}{7cm}
\newcommand{\figureheight}{3cm}

%\usepackage{times}
%\usepackage{helvet}
%\usepackage{bookman}
\usepackage{palatino}

%\newcommand{\captionfont}{\footnotesize}

\usetikzlibrary{calc}

% \newcommand{\SET}[1]  {\ensuremath{\mathcal{#1}}}
% \newcommand{\MAT}[1]  {\ensuremath{\boldsymbol{#1}}}
% \newcommand{\VEC}[1]  {\ensuremath{\boldsymbol{#1}}}
% \newcommand{\Video}{\SET{V}}
% \newcommand{\video}{\VEC{f}}
% \newcommand{\track}{x}
% \newcommand{\Track}{\SET T}
% \newcommand{\LMs}{\SET L}
% \newcommand{\lm}{l}
% \newcommand{\PosE}{\SET P}
% \newcommand{\posE}{\VEC p}
% \newcommand{\negE}{\VEC n}
% \newcommand{\NegE}{\SET N}
% \newcommand{\Occluded}{\SET O}
% \newcommand{\occluded}{o}

\renewcommand{\sfdefault}{lmss}
\sffamily

\newcommand{\listhead}[1] {\textsc{\underline{#1}}}

\definecolor{rouge1}{RGB}{226,0,38}  % red P
\definecolor{orange1}{RGB}{243,154,38}  % orange P
\definecolor{jaune}{RGB}{254,205,27}  % jaune P
\definecolor{blanc}{RGB}{255,255,255} % blanc P

\definecolor{rouge2}{RGB}{230,68,57}  % red S
\definecolor{orange2}{RGB}{236,117,40}  % orange S
\definecolor{taupe}{RGB}{134,113,127} % taupe S
\definecolor{gris}{RGB}{91,94,111} % gris S
\definecolor{bleu1}{RGB}{38,109,131} % bleu S
\definecolor{bleu2}{RGB}{28,50,114} % bleu S
\definecolor{vert1}{RGB}{133,146,66} % vert S
\definecolor{vert3}{RGB}{20,200,66} % vert S
\definecolor{vert2}{RGB}{157,193,7} % vert S
\definecolor{darkyellow}{RGB}{233,165,0}  % orange S
\definecolor{lightgray}{rgb}{0.9,0.9,0.9}
\definecolor{darkgray}{rgb}{0.6,0.6,0.6}

\definecolor{blue900}{HTML}{0D47A1}
\definecolor{blue800}{HTML}{1565C0}

\definecolor{green1}{RGB}{217,255,250}
\definecolor{orange1}{RGB}{250,245,217}

\newcommand{\rcol}[1]{\textcolor{red}{\textit{#1}}}
\newcommand{\gcol}[1]{\textcolor{vert3}{\textit{#1}}}
\newcommand{\bcol}[1]{\textcolor{blue}{\textit{#1}}}
\newcommand{\ycol}[1]{\textcolor{darkyellow}{\textit{#1}}}

\newcommand{\rcolb}[1]{\textcolor{red}{\textit{\textbf{#1}}}}
\newcommand{\gcolb}[1]{\textcolor{vert3}{\textit{\textbf{#1}}}}
\newcommand{\bcolb}[1]{\textcolor{blue}{\textit{\textbf{#1}}}}
\newcommand{\ycolb}[1]{\textcolor{darkyellow}{\textit{\textbf{#1}}}}

\newcommand{\otoprule}{\midrule[\heavyrulewidth]}
\newcommand{\dbacks}[1]{\textbf{\textcolor{red!80!black}{{#1}}}}


\usepackage{tikz,pgfplots}
\pgfplotsset{compat=newest}
\tikzstyle{every picture}+=[remember picture]
\tikzstyle{na} = [baseline=-.5ex]
\everymath{\displaystyle}
\usetikzlibrary{arrows,shapes}
\usetikzlibrary{positioning}

\usepackage{wasysym}

\usepackage{tcolorbox}

\usepackage{algorithm}
\usepackage{algorithmic}

\usepackage{xspace}
\DeclareRobustCommand{\eg}{e.g.,\@\xspace}
\DeclareRobustCommand{\ie}{i.e.,\@\xspace}
\DeclareRobustCommand{\wrt}{w.r.t.\@\xspace}
\DeclareRobustCommand{\wp}{w.p.\@\xspace}

%=============================================
% General commands
%---------------------------------------------
\newcommand{\transpose}[1]{{#1}^\texttt{T}}
\DeclareMathOperator*{\EV}{\mathbb{E}}
\DeclareMathOperator*{\argmax}{\arg\,\max}
\DeclareMathOperator*{\argmin}{\arg\,\min}
\DeclareMathOperator*{\arginf}{\arg\,\inf}
\newcommand{\EVV}[2][x]{\EV_{#1}\left[{#2}\right]}
\newcommand{\norm}[2][\infty]{\left\|#2\right\|_{#1}}
\newcommand{\indfun}[1]{\mathds{1}\left(#1\right)}

\newcommand{\realspace}{\mathbb R}		% realspace
\newcommand{\natspace}{\mathbb N}		% naturalspace
\newcommand{\pdfunc}[1]{p\left(#1\right)}	% density function
\newcommand{\prob}[1]{Pr\left(#1\right)}	% probability
\newcommand{\funoper}[1]{\left[#1\right]}	% function operator
\newcommand{\apx}[1]{\widetilde{#1}}		% approximation symbol
\newcommand{\e}{\boldsymbol{e}} 		% unit vector

\newcommand{\texsub}[1]{\textsc{\tiny #1}}


%=============================================
% RL Commands
%---------------------------------------------
%%% MDPs
\newcommand{\mdp}{\mathcal{M}}				% MDP
\newcommand{\statespace}{\mathcal S}			% state space
\newcommand{\actionspace}{\mathcal A}			% action space
\newcommand{\pmodel}{\mathcal P}			% transition function
\newcommand{\pmodelfun}[1]{\pmodel\left(#1\right)}	% transition function
\newcommand{\rmodel}{\mathcal R}			% reward function
\newcommand{\rmodelfun}[1]{\rmodel\left(#1\right)}	% reward function
\newcommand{\initD}{\mu}					% initial distribution

%% Policy Gradient
\newcommand{\vtheta}{\boldsymbol{\theta}}
\newcommand{\gradJ}[1]{\nabla_{#1}J_\mu(\vtheta)}
\newcommand{\vphi}{\boldsymbol{\phi}}
\newcommand{\DeltaJ}{J(\vtheta') - J(\vtheta)}
\newcommand{\gradApp}[1]{\widehat{\nabla}_{#1}J(\vtheta)}
\newcommand{\gradSim}[1]{\tilde{\nabla}_{#1}J(\vtheta)}
\newcommand{\gradRF}[1]{\hat{\nabla}_{#1}J^{RF}(\vtheta)}
\newcommand{\gradPGT}[1]{\hat{\nabla}_{#1}J^{PGT}(\vtheta)}
\newcommand{\gradGPOMDP}[1]{\hat{\nabla}_{#1}J^{G(PO)MDP}(\vtheta)}
\newcommand{\gradDown}[1]{\underline{\hat{\nabla}_{#1}J}(\vtheta)}
\newcommand{\gradUp}[1]{\overline{\hat{\nabla}_{#1}J}(\vtheta)}

%Specific
\newcommand{\Ets}[2][t]{\mathbb{E}_{#1\vert s}\left[#2\right]}
\newcommand{\Es}[1]{\mathbb{E}_{s}\left[#1\right]}
\newcommand{\Covts}[3][t]{{\mathbb{C}\text{ov}}_{#1\vert s}\left(#2,#3\right)}
\newcommand{\Covs}[2]{{\mathbb{C}\text{ov}}_{s}\left(#1,#2\right)}
\newcommand{\Varts}[2][t]{{\mathbb{V}\text{ar}}_{#1\vert s}\left[#2\right]}
\newcommand{\Vars}[1]{{\mathbb{V}\text{ar}}_{s}\left[#1\right]}
\newcommand{\gradBlack}[1]{\blacktriangledown J(#1)}
\newcommand{\gradIdeal}[1]{\dnabla J(#1)}
\newcommand{\VARRF}{V}
\newcommand{\GRADLOG}{G}
\newcommand{\VARIS}{W}
\newcommand{\HESSLOG}{F}
% short forms 
\newcommand{\wt}[1]{\widetilde{#1}}
\newcommand{\wh}[1]{\widehat{#1}}
\newcommand{\wo}[1]{\overline{#1}}
\newcommand{\wb}[1]{\overline{#1}}

%Colors
\definecolor{crimson}{rgb}{0.86, 0.08, 0.24}
\definecolor{myblue}{rgb}{0.0, 0.44, 1.0}
\definecolor{amber2}{rgb}{1.0, 0.49, 0.0}
\newcommand{\chall}[1]{\textcolor{amber2}{\textbf{#1}}}
\newcommand{\bad}[1]{\textcolor{crimson}{\textbf{#1}}}
\newcommand{\enhance}[1]{\textcolor{myblue}{\textbf{#1}}}

%%%%%%%%%%%%%%%%%%%%%%%%%%%%%%%%%%%%%%%%%%%%%%%%%%%%%%%%%%%%%%%%%%%%%%%%%%%%%%%%
%%%% Some math symbols used in the text
%%%%%%%%%%%%%%%%%%%%%%%%%%%%%%%%%%%%%%%%%%%%%%%%%%%%%%%%%%%%%%%%%%%%%%%%%%%%%%%%

%%%%%%%%%%%%%%%%%%%%%%%%%%%%%%%%%%%%%%%%%%%%%%%%%%%%%%%%%%%%%%%%%%%%%%%%%%%%%%%%
% Multicol Settings
%%%%%%%%%%%%%%%%%%%%%%%%%%%%%%%%%%%%%%%%%%%%%%%%%%%%%%%%%%%%%%%%%%%%%%%%%%%%%%%%
\setlength{\columnsep}{1.5em}
\setlength{\columnseprule}{0mm}

%%%%%%%%%%%%%%%%%%%%%%%%%%%%%%%%%%%%%%%%%%%%%%%%%%%%%%%%%%%%%%%%%%%%%%%%%%%%%%%%
% Save space in lists. Use this after the opening of the list
%%%%%%%%%%%%%%%%%%%%%%%%%%%%%%%%%%%%%%%%%%%%%%%%%%%%%%%%%%%%%%%%%%%%%%%%%%%%%%%%
\newcommand{\compresslist}{%
\setlength{\itemsep}{1pt}%
\setlength{\parskip}{0pt}%
\setlength{\parsep}{0pt}%
}

\captionsetup{justification=raggedright,singlelinecheck=false}

%%%%%%%%%%%%%%%%%%%%%%%%%%%%%%%%%%%%%%%%%%%%%%%%%%%%%%%%%%%%%%%%%%%%%%%%%%%%%%
%%% Begin of Document
%%%%%%%%%%%%%%%%%%%%%%%%%%%%%%%%%%%%%%%%%%%%%%%%%%%%%%%%%%%%%%%%%%%%%%%%%%%%%%

\begin{document}

%%%%%%%%%%%%%%%%%%%%%%%%%%%%%%%%%%%%%%%%%%%%%%%%%%%%%%%%%%%%%%%%%%%%%%%%%%%%%%
%%% Here starts the poster
%%%---------------------------------------------------------------------------
%%% Format it to your taste with the options
%%%%%%%%%%%%%%%%%%%%%%%%%%%%%%%%%%%%%%%%%%%%%%%%%%%%%%%%%%%%%%%%%%%%%%%%%%%%%%
% Define some colors

%\definecolor{lightblue}{cmyk}{0.83,0.24,0,0.12}
\definecolor{lightblue}{rgb}{0.145,0.6666,1}

% % Draw a video
% \newlength{\FSZ}
% \newcommand{\drawvideo}[3]{% [0 0.25 0.5 0.75 1 1.25 1.5]
%    \noindent\pgfmathsetlength{\FSZ}{\linewidth/#2}
%    \begin{tikzpicture}[outer sep=0pt,inner sep=0pt,x=\FSZ,y=\FSZ]
%    \draw[color=lightblue!50!black] (0,0) node[outer sep=0pt,inner sep=0pt,text width=\linewidth,minimum height=0] (video) {\noindent#3};
%    \path [fill=lightblue!50!black,line width=0pt] 
%      (video.north west) rectangle ([yshift=\FSZ] video.north east) 
%     \foreach \x in {1,2,...,#2} {
%       {[rounded corners=0.6] ($(video.north west)+(-0.7,0.8)+(\x,0)$) rectangle +(0.4,-0.6)}
%     }
% ;
%    \path [fill=lightblue!50!black,line width=0pt] 
%      ([yshift=-1\FSZ] video.south west) rectangle (video.south east) 
%     \foreach \x in {1,2,...,#2} {
%       {[rounded corners=0.6] ($(video.south west)+(-0.7,-0.2)+(\x,0)$) rectangle +(0.4,-0.6)}
%     }
% ;
%    \foreach \x in {1,...,#1} {
%      \draw[color=lightblue!50!black] ([xshift=\x\linewidth/#1] video.north west) -- ([xshift=\x\linewidth/#1] video.south west);
%    }
%    \foreach \x in {0,#1} {
%      \draw[color=lightblue!50!black] ([xshift=\x\linewidth/#1,yshift=1\FSZ] video.north west) -- ([xshift=\x\linewidth/#1,yshift=-1\FSZ] video.south west);
%    }
%    \end{tikzpicture}
% }
% 
% \hyphenation{resolution occlusions}
% %%
\begin{poster}%
  % Poster Options
  {
  % Show grid to help with alignment
  columns=4,
  grid=false,
  % Column spacing
  colspacing=1em,
  % Color style
  bgColorOne=white,
  bgColorTwo=white,
  borderColor=blue900,
  headerColorOne=blue800,
  headerColorTwo=blue800,
  headerFontColor=white,
  boxColorOne=white,
  boxColorTwo=lightblue,
  % Format of textbox
  textborder=roundedleft,
  % Format of text header
  eyecatcher=true,
  headerborder=closed,
  headerheight=0.095\textheight,
%  textfont=\sc, An example of changing the text font
  headershape=roundedright,
  headershade=shadelr,
  headerfont=\large\bf\textsc, %Sans Serif
  textfont={\setlength{\parindent}{1.5em}},
  boxshade=plain,
%  background=shade-tb,
  background=plain,
  linewidth=2pt
  }
  % Eye Catcher
  % {\includegraphics[height=9em]{./pics/airlab_logo_reflect.png}} 
  {\includegraphics[height=7.0em]{./pics/polilogo/logoPoliBlue_poster.png}}
%   {\hspace{3.5cm}}
  % Title
  {\bf\textsc{Stochastic Variance-Reduced Policy Gradient}\vspace{0.1em}}
  % Authors
  {\textsc{M. Papini, D. Binaghi, G. Canonaco, M. Pirotta and M. Restelli}\\ 
  {\normalsize \texttt{\{matteo.papini, marcello.restelli\}@polimi.it}},
  {\normalsize \texttt{\{damiano.binaghi, giuseppe.canonaco\}@mail.polimi.it}},  {\normalsize \texttt{\{matteo.pirotta\}@inria.fr}}
  }
  % University logo
  {% The makebox allows the title to flow into the logo, this is a hack because of the L shaped logo.
    %\includegraphics[height=9.0em]{./pics/PoliMI.pdf}%\hspace{.5cm}
    \includegraphics[height=4em]{./pics/inria_sc}
  }

%%%%%%%%%%%%%%%%%%%%%%%%%%%%%%%%%%%%%%%%%%%%%%%%%%%%%%%%%%%%%%%%%%%%%%%%%%%%%%
%%% Now define the boxes that make up the poster
%%%---------------------------------------------------------------------------
%%% Each box has a name and can be placed absolutely or relatively.
%%% The only inconvenience is that you can only specify a relative position 
%%% towards an already declared box. So if you have a box attached to the 
%%% bottom, one to the top and a third one which should be in between, you 
%%% have to specify the top and bottom boxes before you specify the middle 
%%% box.
%%%%%%%%%%%%%%%%%%%%%%%%%%%%%%%%%%%%%%%%%%%%%%%%%%%%%%%%%%%%%%%%%%%%%%%%%%%%%%
    %
    % A coloured circle useful as a bullet with an adjustably strong filling
    \newcommand{\colouredcircle}{%
      \tikz{\useasboundingbox (-0.2em,-0.32em) rectangle(0.2em,0.32em); \draw[draw=black,fill=lightblue,line width=0.03em] (0,0) circle(0.18em);}}

\newcommand{\HL}[1]{\textcolor{blue}{\textbf{#1}}}

%%%%%%%%%%%%%%%%%%%%%%%%%%%%%%%%%%%%%%%%%%%%%%%%%%%%%%%%%%%%%%%%%%%%%%%%%%%%%%
  \headerbox{Motivation}{name=motiv,column=0,row=0,span=1}{
%%%%%%%%%%%%%%%%%%%%%%%%%%%%%%%%%%%%%%%%%%%%%%%%%%%%%%%%%%%%%%%%%%%%%%%%%%%%%%
	\noindent We want to solve \textbf{continuous} Markov Decision Processes (MDPs), \eg robot locomotion
	
	\vspace{2mm}
	
	\noindent\textbf{Policy Gradient} (PG): optimize a parametric policy $\pi_{\vtheta}$ via \textbf{gradient ascent} on performance $J(\vtheta)$.
	
	\vspace{2mm}
	
	\noindent Two main strategies for gradient computation:
	
	\begin{itemize}
	\item \textbf{Full Gradient (FG)}: \bad{sample infefficient} \\$\longrightarrow O(\nicefrac{N}{\epsilon})$
	\item \textbf{Stochastic Gradient (SG)}: \bad{slow convergence} \\$\longrightarrow O(\nicefrac{1}{\epsilon^2})$
	\end{itemize}

	\noindent Sample efficiency is crucial in Reinforcement Learning (RL), where collecting samples is extremely expensive $\implies$ \textbf{FG} often unfeasible.

	\vspace{2mm}

	\noindent Slower convergence of \textbf{SG} due to \bad{gradient variance}.
	\\A solution from the Supervised Learning (SL) literature (finite-sum optimization):
	\begin{itemize}
		\item \textbf{Stochastic Variance-Reduced Gradient (SVRG)} $\longrightarrow O(N+\nicefrac{N^{\nicefrac{2}{3}}}{\epsilon})$
	\end{itemize}
}


%%%%%%%%%%%%%%%%%%%%%%%%%%%%%%%%%%%%%%%%%%%%%%%%%%%%%%%%%%%%%%%%%%%%%%%%%%%%%%
\headerbox{Problem}{name=problem,column=0,row=1, span=1,below=motiv}{
	%%%%%%%%%%%%%%%%%%%%%%%%%%%%%%%%%%%%%%%%%%%%%%%%%%%%%%%%%%%%%%%%%%%%%%%%%%%%%%
	\noindent Can we apply \textbf{SVRG} to policy optimization?\\
	Compared to SL, RL has three additional challenges:
	\begin{enumerate}
		\item \chall{Non-concavity}: policy performance $J(\vtheta)$ is typically a non-concave objective
		\item \chall{Infinite dataset}: expectation over all possible trajectories cannot be expressed as a finite sum  
		\item \chall{Non-stationarity}: the data-generating distribution changes as we learn
	\end{enumerate}
}


\headerbox{Related Work}{name=related,column=0,row=1, span=1,below=problem,above=bottom}{
	%%%%%%%%%%%%%%%%%%%%%%%%%%%%%%%%%%%%%%%%%%%%%%%%%%%%%%%%%%%%%%%%%%%%%%%%%%%%%%
	\noindent From the \textbf{SL} literature, \textit{separate} study of: 
	\begin{itemize}
		\item \chall{Non-concavity}~\citep{du2017svrgpe,reddi2016stochastic}
		\item \chall{Infinite dataset}~\citep{harikandeh2015stopwasting,bietti2017stochastic}
	\end{itemize}

	\noindent From the \textbf{RL} literature:
	\begin{itemize}
		\item \textbf{SVRPGE}~\citep{du2017svrgpe}: SVRG for policy \textit{evaluation}
		\item \textbf{SVRPO}~\cite{xu2017svrgtrpo}: direct application of SVRG to TRPO
	\end{itemize}

\chall{Nonstationarity} never addressed explicitly!
}

%%%%%%%%%%%%%%%%%%%%%%%%%%%%%%%%%%%%%%%%%%%%%%%%%%%%%%%%%%%%%%%%%%%%%%%%%%%%%%
\headerbox{Contributions}{name=contrib,column=1,row=0, span=2}{
	%%%%%%%%%%%%%%%%%%%%%%%%%%%%%%%%%%%%%%%%%%%%%%%%%%%%%%%%%%%%%%%%%%%%%%%%%%%%%%
	\begin{itemize}
		\item We design \enhance{SVRPG}, a variant of SVRG for the PG framework that explicitly addresses the \chall{challenges} of RL.
		\item We study the \textbf{convergence} properties of SVRPG
		\item  We devise \textbf{heuristics} for meta-parameter selection in order to develop a practical algorithm.
		\item We evaluate SVRPG on simulated \textbf{continuous control tasks}. 
	\end{itemize}
}


%%%%%%%%%%%%%%%%%%%%%%%%%%%%%%%%%%%%%%%%%%%%%%%%%%%%%%%%%%%%%%%%%%%%%%%%%%%%%%
  \headerbox{SVRPG}{name=svrpg,column=1,span=2,row=1,below=contrib}{
%%%%%%%%%%%%%%%%%%%%%%%%%%%%%%%%%%%%%%%%%%%%%%%%%%%%%%%%%%%%%%%%%%%%%%%%%%%%%%
\begin{itemize}
	\item Take any \textbf{unbiased} policy gradient estimator $g(\tau|\vtheta)$ (\eg REINFORCE)
	\item \textbf{SVRG} idea: combine frequent, inaccurate \textbf{SG} estimations with rare, accurate \textbf{FG} computations.
	\item Manage \chall{non-concavity} with \textbf{smooth} policies (\eg Gaussian)
	\item Approximate \chall{infinite dataset} with a large \textbf{batch size} $N$
	\item Correct \chall{nonstationarity} with \textbf{importance weighting}
	\item At each \textbf{iteration}, update policy parameter as $\vtheta\gets\vtheta+\alpha\gradBlack{\vtheta}$ with the following \enhance{SVRPG estimator}:	
\end{itemize}
%
\vspace{1cm}
%
\begin{center}
	\tikz[overlay]{
		\node[draw=myblue, inner sep=20pt, ultra thick] {
			\Large{$
			\blacktriangledown J(\vtheta) = \wh{\nabla}_N J(\wt{\vtheta}) + \wh{\nabla}_B J({\vtheta}) - \omega(\vtheta, \wt{\vtheta}) \wh{\nabla}_B J({\wt{\vtheta}})
			$}
		};
	}
\end{center} 
%
\vspace{1.5cm}
\fbox{
\begin{minipage}[t]{.3\textwidth}
\begin{center}
	\textbf{FG term}
	
	\begin{equation*}
	\frac{1}{N}\sum_{i=1}^{N} g(\tau_i|\wt{\vtheta})
	\end{equation*}\\
	
	\begin{itemize}
		\item Trajectories sampled with \textbf{snapshot} policy $\pi_{\wt{\vtheta}}$
		\item Accurate FG approximation with $N$ samples \\$\longrightarrow$ \bad{expensive!}
		\item Updated once every $m$ iterations (one \textbf{epoch})
		to save sample efficiency
	\end{itemize}
\end{center}
\end{minipage}
}
%
\hfill\fbox{
\begin{minipage}[t]{.3\textwidth}
	\begin{center}
		\textbf{SG term}
		
		\begin{equation*}
		\frac{1}{B}\sum_{i=1}^{B} g(\tau_i|\vtheta)
		\end{equation*}
		
		\begin{itemize}
			\item Trajectories sampled with current policy $\pi_{\vtheta}$ (\textbf{online})
			\item Gradient estimation with $B<<N$ samples\\
			$\longrightarrow$ \bad{high variance!}
			\item Stabilized by the low-variance FG term
		\end{itemize}
	\end{center}
\end{minipage}
}
%
\hfill\fbox{
\begin{minipage}[t]{.25\textwidth}
	\begin{center}
			\textbf{Correction term}
			
			\begin{equation*}
			\frac{1}{B}\sum_{i=1}^{B}\underbrace{\omega(\tau_i|\vtheta,\wt{\vtheta})}_{\frac{p(\tau_i|\wt{\vtheta})}{p(\tau_i|\vtheta)}} g(\tau_i|\vtheta)
			\end{equation*}
			
			\begin{itemize}
				\item Same trajectories as the SG term
				\item But evaluated in the snapshot parameter!
				\item \textbf{Importance weighting}
		\end{itemize}
	\end{center}
\end{minipage}
}
%
}

\headerbox{Convergence}{name=convergence,column=1,span=2,row=2, below=svrpg, above=bottom}{
\begin{align*}
&\EVV[]
{\norm[2]{\nabla J(\vtheta_A)}^2} 
\leq
\frac{J(\vtheta^*)-J(\vtheta_0)}{\psi T} +
\frac{\zeta}{N}
+\frac{\xi}{B},
\end{align*}

Serve piu' spazio, ma l'organizzazione del box sopra e' tutta da vedere. Mancano anche le due basic properties
}

\headerbox{Heuristics}{name=heuristics,column=3,span=1,row=0,height=0.5,height=0.32}{
	\begin{itemize}
		\item Adaptive step size
		\item Adaptive epoch size
		\item Normalized Importance Weighting
		\item Critic (baseline)
	\end{itemize}
}

%%%%%%%%%%%%%%%%%%%%%%%%%%%%%%%%%%%%%%%%%%%%%%%%%%%%%%%%%%%%%%%%%%%%%%%%%%%%%%
\headerbox{Empirical Results}{name=emp,column=3,row=1,span=1,below=heuristics,height=0.45}{
	Direi che ci stanno 2 grafici, ma devo pensare a quali mettere 
	
	Eventualmente saltano Related Work e le References
}

\headerbox{References}{name=ref,column=3,row=1,span=1,below=emp,above=bottom}{
        \tiny
\bibliographystyle{plainnat}
\begingroup
\renewcommand{\section}[2]{}%
\bibliography{../svrpg.bib}
\endgroup
}


\end{poster}



\end{document}
